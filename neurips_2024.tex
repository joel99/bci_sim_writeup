\documentclass{article}

\usepackage[preprint]{neurips_2024}
\usepackage[utf8]{inputenc}
\usepackage[T1]{fontenc}
\usepackage{hyperref}
\usepackage{url}
\usepackage{amsfonts}
\usepackage{microtype}


\title{A NeuroAI-derived Brain-Computer Interface Simulator}

\author{%
  Joel Ye \\
  Carnegie Mellon University\\
  \texttt{joelye@andrew.cmu.edu} \\
  \And
  Chris Ki \\
  Carnegie Mellon University\\
  \texttt{cski@andrew.cmu.edu} \\
}


\begin{document}


\maketitle


\begin{abstract}
  Brain-computer interfaces (BCIs) are a promising technology for enabling communication and control of computers using the brain. However, the development of BCIs is currently limited by the expense and inaccessibility of real world experiments, which motivates the development of a simulation platform to evaluate BCI algorithms. Here, we explore a strategy for simulating BCI adaptation and control through a deep network controller trained with deep RL. Critically, our method avoids any reliance on pre-existing neural datasets, allowing for the evaluation of BCI algorithms in a fully synthetic environment.
  TODO RESULTS statement.
  TODO CONCLUSION statement.

\end{abstract}


\section{Introduction}

P1.
Intracortical brain-computer interfaces (iBCIs) are an emerging technology where electrodes implanted in the motor cortex record neural activity that can be used to decode a user's movement intention. BCI control can be achieved even with simple linear decoders, given the remarkably strong physiological connections between motor cortical activity and movement. Naturally, over two decades of research, researchers have proposed many decoders that achieve higher predictive performance on offline, static datasets, but evolution in the decoders used for online control has been much slower. The major reason for this disparity is the inaccessibility and expense of real-world experiments. Implanted BCIs are restricted to pre-clinical or clinical research and are thus studied in only a few dozen institutions worldwide. In each of these institutions, experiments work with one or two users for only a few hours a week while coordinating many different research priorities.
These decoders are typically trained on a dataset of neural activity and
BCI technology is the subject of a number of ongoing clinical trials (e.g. with Neuralink and Paradromics), as its invasive nature is justified by its demonstrated high performance potential. Nonetheless, this performance
Must conclude by talking about prohibitve expense and need for simulation.

P2.
Simulation goals and limitations. Mention of Sim2Real, that's not what we're doing in near future. Just want to eval one slice of the BCI problem, user adaptation.
Liang et al line of work review. Limitations, and need for data independency for full synthetic eval.

P3. Overview of conceptual motivation and approach. Low-D. (Related work begins here).

\section{Methods}

Content goes here.

\section{Results}

Content goes here.

\section{Discussion}

Content goes here.

\section*{References}

{
\small

[1] Author, A. (2024) Paper title. {\it Conference Name}.

[2] Author, B. (2024) Another paper title. {\it Journal Name}.

}

\end{document}